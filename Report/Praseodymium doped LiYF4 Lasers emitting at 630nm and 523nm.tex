\documentclass[conference]{IEEEtran}
\IEEEoverridecommandlockouts
%The preceding line is only needed to identify funding in the first footnote. If that is unneeded, please comment it out.


%\usepackage{cite}
\usepackage{amsmath,amssymb,amsfonts}
\usepackage{algorithmic}
\usepackage{graphicx}
\usepackage{textcomp}
\usepackage{xcolor}

% JH/zusätzliche Packages, die ich verwendet habe
\usepackage{hyperref}


% JH/Definition eigener Befehle
\newcommand{\R}{\mathbb{R}}
\newcommand{\N}{\mathbb{N}}
\newcommand{\TSShoch}{\mathhbb\textsuperscript{}}
%\textsuperscript{2}

\def\BibTeX{{\rm B\kern-.05em{\sc i\kern-.025em b}\kern-.08em
    T\kern-.1667em\lower.7ex\hbox{E}\kern-.125emX}}
    
    
\usepackage[backend=bibtex, style=numeric]  
           {biblatex}
\addbibresource{Literature.bib}   
    
\begin{document}

\title{Diode pumped $Pr^{3+}$:LiYF4 lasers emitting at 640nm and 523nm\\
}

\author{Niklas H.}

\maketitle

\begin{abstract}
In this report, a Praseodymium doped LiYF4 (Pr:Ylf) laser consisting of a hemispheric cavity of 50mm length and a 6mm long longitudinally pumped crystal (0.8\% doping) is reported. Depending on the mirror set used, the 640nm ($^3P_0$ to $^3F_2$) and 523nm ($^3P_0$ to $^3H_5$) lines of the $Pr^{3+}$-Ion can be amplified. For the red line, an OC with 1.8\%T resulted in X.
\end{abstract}

\begin{IEEEkeywords}
LiYF4 Laser, DPSS Laser, Pr:Ylf Laser
\end{IEEEkeywords}
%-------------------------------------------------------------------------------------
\section{Introduction}
%-------------------------------------------------------------------------------------

Since the introduction of Lasers in 1960 by Theodore Maiman, many gain media for obtaining optical gain and therefore laser action have been introduced. One these media is LiYF4 doped with $Pr^{3+}$-Ions (Pr:Ylf), which is able to operate at multiple lines, e.g. 720nm, 607nm, 640nm and 523nm, which is why special research efforts have been directed towards this material. For this report, the 640nm ($^3P_0$ to $^3F_2$) and 523nm ($^3P_0$ to $^3H_5$) transitions are of relevance. The $Pr^{3+}$-Ion reaches the excited state ($^3P_0$) by being excited to the $^3P_2$ state and fastly relaxing to $^3P_0$. The excitation process is performed most efficiently by using 444nm pumplight. Since InGaN laser diodes emit around this exact wavelength and are commercially available at several watts of optical power, they have been used extensively for pumping Pr:Ylf lasers [Sources]. For linear end-pumped resonator designs using InGaN laser diodes, slope efficiencies of up to 49\% for 523nm and 57\% for 640nm have been reported \cite{Luo.2016}. Apart from 

\section{Experimental setup}
\section{Results}
\section{Conclusion}
%-------------------------------------------------------------------------------------
%-------------------------------------------------------------------------------------
%\section*{References}
\printbibliography
%-------------------------------------------------------------------------------------
%-------------------------------------------------------------------------------------
\end{document}
